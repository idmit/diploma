\chapter{Результаты и выводы}	% Заголовок

\section*{Результаты}

В ходе работы был создан расширяемый программный пакет с возможностью экспорта результатов в совместимые инструменты. С его помощью было построено 243 аппроксимации копулы (для каждой пары параметров и их значений), каждая из которых была визуализирована для анализа зависимостей между вероятностями попадания параметров в оценённые интервалы.

При условии безупречной работы алгоритма интервального оценивания, все графики копул соответствовали бы плотности копулы $\Pi$. Это бы свидетельствовало о том, что попадание в компакт или же не попадание в него по одному параметру, не влияет на попадания по другим параметрам. Но исследование показало, что это не так.

Между вероятностями попадания параметров присутствуют стохастические связи. Это справедливо только для интервальных оценок, доставляемых алгоритмом с вектором параметров $\xi$. На основе плотностей копул, полученных в этой работе, следует отобрать пары параметров и значений, на которые следует обращать внимание при настройке алгоритма оценивания. Возможно, удастся добиться построения таких оценок, для которых вероятности попадания будут независимы.

\section*{Выводы}

В качестве пары параметров для анализа выберем, например, параметры $\sigma$ и $x_m$.
\begin{figure}[h]
	{\input{plot_10_20_1000}}
	\caption{ Плотность распределения указанной копулы}
	\label{fig:smallxm}
\end{figure}
На рисунке \ref{fig:smallxm} приведена плотность копулы, соответствующей оценкам вероятностей попадания параметров $\sigma$ и $x_m$ в полученные для них интервалы. Пик в левом верхнем углу контурного графика наталкивает на мысль, что нужно провести исследование того подмножества модельных наборов параметров, который приводит к такой картине. Это прямое применение полученных аппроксимаций копул.

\begin{figure}[h]
	{\input{ZY}}
	\caption{Плотности распределения указанных процессов при фиксированном $t$}
	\label{fig:zy}
\end{figure}
На рисунке синим изображена плотность распределения Y(t). При увеличении модельного значения $x_m$ она сдвигается вправо, отдаляясь от моды распределения $Z(t)$. Это позволяет алгоритму оценивания доставлять лучшие результаты.

\begin{figure}[h]
	{\input{plot_10_21_1000}}
	\caption{Плотность распределения указанной копулы}
	\label{fig:bigxm}
\end{figure}
На рисунке \ref{fig:bigxm} приведена плотность копулы, соответствующей тем же оценкам, но с большим значением $x_m$. Пик в правом верхнем углу означает, что алгоритм интервального оценивания успешно справляется со своей задачей, когда модельное значение $x_m$ увеличивается. Это подтверждает априорные предположения, представленные рисунком \ref{fig:zy}. Можно построить количественную меру, показывающую при каком соотношении модельных значений $x_m$ и $\sigma$, алгоритм интервального оценивания улучшает свою работу.

Таким образом
\begin{itemize}
  \item Форма копулы подтвердила априорные предположения о характере связи оценок $\sigma$ и $x_m$;
  \item При малом модельном значении $x_m$ копула продемонстрировала наличие неизвестной ранее связи между этими оценками.
\end{itemize}

\clearpage
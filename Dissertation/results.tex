\chapter{Результаты и выводы}	% Заголовок

\section*{Результаты}

В ходе работы был создан расширяемый программный пакет. Он масштабируем в смысле расширения его функциональных возможностей (поддерживаемых классов распределений, моделей копул, форматов обмена данными). С его помощью было построено 243 аппроксимации копулы (для каждой пары параметров и их значений). Каждая такая оценка была визуализирована для анализа качественных и количественных характеристик результатов интервального оценивания, полученных композицией алгоритмов из работы [Сидоровская, 2015].

В качестве асимптотической оценки таких двумерных распределений, при условии, что алгоритмы интервального оценивания обеспечивают состоятельные оценки границ компактов, ожидается получить плотности копулы $\Pi$. Это было бы свидетельством того факта, что попадание или не попадание модельного значения одного параметра в его компакт никак не связано с аналогичными исходами интервального оценивания по другим параметрам. Но исследование показало, что это не так.

Между исходами интервального оценивания границ компактов присутствуют стохастические связи.
%% опять речь о $\xi$, который нигде выше формально не определили
Это справедливо только для интервальных оценок, доставляемых алгоритмом с вектором параметров $\overline{\xi}$. На основе оценок плотностей копул, полученных в этой работе, следует отобрать пары параметров и значений, на которые следует обращать внимание при последующей работе с алгоритмом оценивания. Возможно, удастся добиться построения таких оценок, для которых вероятности попадания будут независимы.


\section*{Выводы}

В качестве примера выберем для анализа пару параметров $\sigma$ и $x_m$.
\begin{figure}[h]
	{\input{plot_11_40_1000}}
	\caption{ Плотность распределения указанной копулы}
	\label{fig:smallxm}
\end{figure}
На рисунке \ref{fig:smallxm} приведена плотность копулы, соответствующей оценкам вероятностей попадания модельных значений параметров $\sigma$ и $x_m$ в построенные для них интервалы.
Максимум в нижней части контурного графика наталкивает на мысль, что нужно провести исследование того подмножества модельных наборов параметров, который приводит к такой картине. Это всего лишь одно из непосредственных возможных применений построенных оценок в форме копул.

\begin{figure}[H]
	\centering
	\includegraphics[width=0.6\linewidth]{ZY.png}
	\caption{Плотности распределения указанных процессов при фиксированном $t$}
	\label{fig:zy}
\end{figure}
На рисунке \ref{fig:zy} синим изображена плотность распределения Y(t). При увеличении модельного значения $x_m$ она сдвигается вправо, отдаляясь от моды распределения $Z(t)$. С теоретической точки зрения, это должно обеспечивать условия, при которых вероятность построения границ компактов будет выше, алгоритм интервального оценивания будет доставлять лучшие результаты.

\begin{figure}[H]
	{\input{plot_11_41_1000}}
	\caption{Плотность распределения указанной копулы}
	\label{fig:bigxm}
\end{figure}
На рисунке \ref{fig:bigxm} приведена плотность копулы, соответствующей тем же оценкам, но с большим значением $x_m$. Максимум в верхней правой части означает, что алгоритм интервального оценивания успешно справляется со своей задачей, когда модельное значение $x_m$ увеличивается. Это подтверждает априорные теоретически обоснованные предположения, иллюстрируемые на \ref{fig:zy}. Можно построить вероятностную меру, количественно оценивающую связь между соотношением  модельных значений $x_m$ и $\sigma$ и вероятностью успеха  алгоритма интервального оценивания.

Таким образом
\begin{itemize}
  \item Форма копулы подтвердила априорные предположения о характере связи оценок $\sigma$ и $x_m$;
  \item Локальный экстремум в этой части носителя меры копулы  (при малом модельном значении $x_m$) даёт формально обоснованный повод для проведения дальнейших исследований алгоритма построения границ компакта, причём копула обеспечивает нас информацией о том, на каком подмножестве пар значений параметров сигма и $x_m$ следует сосредоточить внимание.
\end{itemize}

Рассмотрим теперь пару $\sigma$, $\lambda$.
\begin{figure}[H]
	{\input{plot_10_20_1000}}
	\caption{Плотность распределения указанной копулы}
	\label{fig:smalllambda}
\end{figure}
Параметр $\lambda$ определяет интенсивность потока пуассоновских событий. На рисунке \ref{fig:smalllambda} увеличение значения оценки копулы при вероятности правильной оценки $\lambda$, близкой к 1, можно интерпретировать как успешную идентификацию всех выбросов. А значит выборка для оценки границ компакта для $x_m$ построена успешно.

Увеличение значения оценки копулы при вероятности правильной оценки $\lambda$, близкой к 0, сейчас откровенно озадачивает.

\begin{figure}[H]
	{\input{plot_10_21_1000}}
	\caption{Плотность распределения указанной копулы}
	\label{fig:biglambda}
\end{figure}
Копула на рисунке \ref{fig:biglambda} также соответствует модели процесса $X(t)$.

Таким образом форма копулы также подтвердила априорные предположения о характере связи оценок $\sigma$ и $\lambda$.

\clearpage

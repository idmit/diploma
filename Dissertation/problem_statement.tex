\chapter{Постановка задачи}	% Заголовок

Исследуемый в работе процесс, ординаты которого доступны для непосредственного измерения, представляет собой реализацию случайного процесса следующего вида:
\begin{equation}\label{eq:model}
X(t) = Z(t) + Y(t)
\end{equation}
Здесь компонента $Z(t)$ характеризует равновесное состояние системы и имеет вид
\begin{equation}\label{eq:mod_wiener}
Z(t) = \mu + \sigma W(t),
\end{equation}
где
$\mu$, $\sigma$ --- константы; $W(t)$ --- винеровский случайный процесс.

Вторая компонента $Y(t)$ описывает внешние возмущения, привносимые в систему в моменты времени с нулевой мерой Лебега, и представляет из себя пуассоновский поток событий следующего вида:
\begin{equation}\label{eq:mod_markovsky}
Y(t) = \sum_{j = 0}^K P_j \cdot I(t - \tau_j) \cdot e^{-\lambda_c(t - \tau_j)},
\end{equation}
где фигурируют следующие параметры:
\begin{itemize}
\item $K$~---~число событий, поступивших до момента $t$;
\item $\tau_j$~---~время возникновения $j$-ого возмущения;
\item $I$~---~функция Хевисайда: $I(x) = \begin{cases}
0, &x < 0, \\
1, &x \geq 0;
\end{cases}$
\item $\lambda_c$~---~константа, одинаковая для всех событий;
\item $P_j$~---~случайная величина, распределённая по закону Парето с параметрами $x_m$ и $\alpha$, одинаковыми для всех событий:
\[ P_j \sim Pareto(x_m, \alpha) \]
\[
f_{P_j}(x) = \begin{cases}
\frac{\alpha x_m^\alpha}{x^{\alpha+1} }, &x \geqslant x_m, \\
0, &x < x_m.
\end{cases}
\]
\end{itemize}
Как видно, величина возмущения распределена по степенному закону или закону с тяжёлым хвостом.
Интервал времени между моментами возникновения возмущений --- случайная величина, распределённая по экспоненциальному закону с параметром $\lambda$:
\begin{gather}
\Delta \tau_j = t_j - t_{j-1} \\
\Delta \tau_j \sim Exp(\lambda).
\end{gather}

Полагается, что известны результаты наблюдений ординат случайного процесса $X(t)$, осуществляемых через равные промежутки времени:
\begin{equation}
\{ t_i, X_i \}_{i=0}^N, \; X_i = X(t_i), \; t_i = i\frac{T}{N},
\end{equation}
где
$T$ --- промежуток времени наблюдения за $X(t)$, а $N$ --- число разбиений промежутка времени.

Требуется оценить параметры предложенной математической модели, а именно $\mu$, $\sigma$, $x_m$, $\alpha$, $\lambda$, $\lambda_c$, на основе реализаций траекторий процесса $X(t)$. В рамках магистерской работы Сидоровской (2015 г.) был реализован алгоритм оценивания параметров этого процесса. Из-за сложности этого процесса, стандартные методы оценки параметров не использовались:
\begin{itemize}
  \item Процесс не стационарен, поэтому метод моментов не применим;
  \item Для использования метода максимального правдоподобия предварительно требуется оценивать границы компакта, например, методами класса Монте-Карло, и только затем применять метод максимального правдоподобия. В работе Сидоровской алгоритм нахождения точечных оценок также работает в два этапа, но методы класса Монте-Карло (Particle Filters) используются на обоих этапах.
\end{itemize}

В рамках применённого подхода было выяснено, что наиболее критичным этапом является этап интервального оценивания. Отвечающий за него алгоритм представляет собой композицию нескольких алгоритмов, имеет вектор параметров $\overline{\xi}$ и на основе реализации траектории случайного процесса доставляет интервальные оценки его параметров $\mu$, $\sigma$, $x_m$, $\alpha$, $\lambda$, $\lambda_c$.

В ходе вычислительных экспериментов на модельных данных было получено, что доставляемые интервальные оценки верны для всей шестёрки параметров процесса в малом проценте случаев. Поэтому требуется проанализировать результаты оценивания границ компакта, которому принадлежат искомые оценки параметров модели, а именно $\mu$, $\sigma$, $x_m$, $\alpha$, $\lambda$, $\lambda_c$, используя данные результаты экспериментов Сидоровской.

Указанные эксперименты проводились при фиксированном $\overline{\xi}$, для 144 различных шестёрок $\mu$, $\sigma$, $x_m$, $\alpha$, $\lambda$, $\lambda_c$. Для каждой шестёрки моделировались траектории процесса длиной 100, 250 и 1000 элементов, по 100 раз для каждой длины. Таким образом набор данных представляет собой 143 200 шестёрок границ отрезков.

\clearpage
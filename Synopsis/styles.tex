%%% Макет страницы %%%
\oddsidemargin=-13pt
\topmargin=-66pt
\headheight=12pt
\headsep=38pt
\textheight=732pt
\textwidth=484pt
\marginparsep=14pt
\marginparwidth=43pt
\footskip=14pt
\marginparpush=7pt
\hoffset=0pt
\voffset=0pt
%\paperwidth=597pt
%\paperheight=845pt
\parindent=1.5cm                  % Размер табуляции (для красной строки) в начале каждого абзаца
\renewcommand{\baselinestretch}{1.25}
\newfloat{scheme}{tb}{sch}

\newcommand{\tr}{\mathrm{Tr}}
\newcommand{\e}{\mathrm{e}}
\renewcommand{\imath}{\mathrm{i}}

\newcommand{\N}{\mathbb{N}}
\newcommand{\Z}{\mathbb{Z}}
\newcommand{\R}{\mathbb{R}}
\renewcommand{\C}{\mathbb{C}}

\renewcommand{\leq}{\leqslant}
\renewcommand{\geq}{\geqslant}

\renewcommand{\Re}{\mathop{\mathrm{Re}}\nolimits}
\renewcommand{\Im}{\mathop{\mathrm{Im}}\nolimits}

 \renewcommand{\d}{\mathrm{d}}

\newcommand{\dd}[2]{\frac{\d{#1}}{\d{#2}}}
\newcommand{\ddd}[2]{\frac{\d^2{#1}}{\d{#2}^2}}
\newcommand{\pd}[2]{\frac{\partial{#1}}{\partial{#2}}}
\newcommand{\pdd}[2]{\frac{\partial^2{#1}}{\partial{#2}^2}}
\newcommand{\pdp}[3]{\frac{\partial^2{#1}}{\partial{#2}\partial{#3}}}
\newcommand{\pdet}[1]{\det\left({#1}\right)}
\renewcommand{\exp}[1]{\mathrm{e}^{#1}}
\newcommand{\half}[1]{\frac{#1}{2}}

%%% Кодировки и шрифты %%%
\ifxetex
  \setmainlanguage{russian}
  \setotherlanguage{english}
  \defaultfontfeatures{Ligatures=TeX,Mapping=tex-text}
  \setmainfont{Times New Roman}
  \newfontfamily\cyrillicfont{Times New Roman}
  \setsansfont{Arial}
  \newfontfamily\cyrillicfontsf{Arial}
  \setmonofont{Courier New}
  \newfontfamily\cyrillicfonttt{Courier New}
\else
  \IfFileExists{pscyr.sty}{\renewcommand{\rmdefault}{ftm}}{}
\fi

%%% Выравнивание и переносы %%%
\sloppy
\clubpenalty=10000
\widowpenalty=10000

%%% Библиография %%%
\makeatletter
\bibliographystyle{utf8gost71u}   % Оформляем библиографию по ГОСТ 7.1 (ГОСТ Р 7.0.11-2011, 5.6.7)
\renewcommand{\@biblabel}[1]{#1.} % Заменяем библиографию с квадратных скобок на точку
\makeatother

%%% Изображения %%%
\graphicspath{{images/}}          % Пути к изображениям

%%% Шаблон %%%
\newcommand{\todo}[1]{\textcolor{red}{#1}}